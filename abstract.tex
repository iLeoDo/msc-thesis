\begin{abstract}
As the development of Internet, the total amount of data is exponentially increasing. However, the Semantic Web has not been mature, so the implied data of webpages in the traditional World Wide Web cannot be presented in a structured way. Therefore, it appears particularly important to filtering out the target webpage we need and extracting structured information among billions of webpages. 

This project uses the seminar announcement webpages in University of Oxford as a case study and gives a new pipeline to implement the solution. This solution is based on Ontologies and combined the techniques of Machine Learning and Visual Analytics to web data analysis and extraction. We implemented an intelligent crawler to get all the webpages on the University of Oxford's website. An active training classifier is used to filter out seminar announcement pages among them. Users can make correction through visual analytics to interact with the classifier. Then, we implemented a semi-automatic information extraction system. With the help of extraction ontology, users can use visual analytics to easily extract target information. Finally, a well designed visualisation is used to display the extracted seminar announcements, which makes the solution more complete and more usable. Furthermore, the introduction of Ontologies improves the system's ability to extend and migrate.

We got satisfying test results in our experiment as well. Though there is no ready-made dataset, we created one by ourselves with manual classification. Our filter reached a rate of 85\% correctness.

Generally speaking, for text analysis in this information explosion era, this kind of solution provides us a new entry point.

\end{abstract}
